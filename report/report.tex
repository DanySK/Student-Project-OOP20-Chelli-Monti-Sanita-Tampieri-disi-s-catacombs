\documentclass[a4paper,12pt]{report}

\usepackage[italian]{babel}
\usepackage[utf8]{inputenc}
\usepackage[T1]{fontenc}
\usepackage{pgf-umlcd}
\usepackage[style=numeric-comp]{biblatex}

\addbibresource{bibliografia.bib}
\title{DISI's catacomb report}
\author{Chelli M., Monti G., Sanità R., Tampieri E.}

\begin{document}
    \maketitle
    \tableofcontents
    \chapter{Analisi}
    \section{Requisiti}
    \subsection{Requisiti funzionali}
    % Qui si meotte la parte di analisi
    \par Il team si pone l'obiettivo di realizzare un gioco 2D roguelike,
    ovvero caratterizzato dall'esplorazione di livelli generati proceduralmente
    di un dungeon, da un gameplay a turni, da grafica tile-based e la morte
    permanente del giocatore \cite{wiki:Roguelike}, simile ai giochi Enter the
    Gungeon e Nuclear Throne.
    \par Il giocatore dovrà affrontare una serie di piani composti da stanze contenenti
    nemici e oggetti fino ad arrivare a un boss finale. La mappa di gioco sarà generata
    in modo casuale e sarà casuale anche la stanza in cui il personaggio comincerà
    la sua avventura.
    \par Inizialmente il protagonista avrà 5 vite, un armatura al 100\%
    e un arma base che potrà cambiare con armi più avanzate che troverà nel corso
    della partità. Ad ogni colpo subito il personaggio perderà una percentuale dell'
    armatura nel momento in cui l'armatura raggiungerà la percentuale dello 0\%
    il danno subito toglierà una vita al personaggio, se il personaggio perde tutte
    e 5 le vite il giocatore ha perso e il gioco termina.
    \par Il gioco può terminare anche se viene sconfitto il boss della mappa ovvero
    un nemico di dimensione maggiore rispetto a quelli incontrati nelle stanze precedenti.
    Tale boss possiederà anche abilità superiori e armi letali per il personaggio.
    \par Una volta terminato il gioco è possibile scegliere se ricominciare la partita
    o recuperare un savepoint precedente.
    \par Se il giocatore si trova in difficoltà può trovare sparsi per la mappa di gioco
    alcune vite o armature che lo aiuteranno a sopravvivere fino alla battaglia finale.
    \par Esistono 2 tipi di nemici base:
    \begin{itemize}
        \item Nemici che attaccano da lontano
        \item Nemici che attaccano da vicino
    \end{itemize}
    \par Il gioco si basa sull'abilità del giocatore ma anche su una percentuale di
    fortuna nel trovare vite, armature ed armi che lo aiuteranno a sconfiggere il
    boss finale.
    \subsection{Requisiti non funzionali}
    \begin{itemize}
        \item Il gioco dovrà risultare fluido e reattivo anche su macchine con hardware non recenti..
        \item Il gioco dovrà avere una grafica e comandi chiari e intuitivi.
    \end{itemize}
    \section{Analisi e modello del dominio}
    % modello del dom
    \par Il sistema gestisce la generazione delle mappe con le varie stanze e le interazioni tra il
    personaggio e i nemici.
    \par Oltre ai nemici il personaggio potrà interagire con oggetti trovati nelle varie stanze.
    \par Il personaggio possiede principalmente un numero di vite che aumenteranno con gli oggetti
    o diminuiranno se colpiti da un nemico e una abilità speciale con un tempo di recupero variabile a seconda degli oggetti.
    \par I nemici potranno essere di due tipi principali, melee (che attaccano da vicino) e ranged (che attaccano da lontano)
    e si muoveranno all'interno delle stanze e avranno una vita massima. Nemici più difficili da sconfiggere,
    come i Boss avranno una maggiore vita massima e mosse speciali che useranno in determinati step durante il combattimento.
    \par La mappa è strutturata in più livelli. Ogni livello è costituito da varie stanze stanze dove compariranno nemnici,
    NPC ("Non Playable Character") e/o oggetti utilizzabili dal giocatore. Al termine di ogni stanza il giocatore si potrà
    spostare a quelle adiacenti o esplorare la mappa. Una sola stanza permetterà di passare al piano successivo.
    Arrivato all'ultimo paino in personaggio dovrà affrontare il Boss e dopo il combattimento il gioco terminerà.
    \par \par Gli elementi considerati nel modello sono sintetizzati in Figura 1.1


    \chapter{Design}
    \section{Architettura}
    \section{Design dettagliato}
    \chapter{Sviluppo}
    \section{Testing automatizzato}
    \section{Metodologia di lavoro}
    \section{Note di sviluppo}
    \chapter{Commenti finali}
    \section{Autovalutazione e lavori futuri}
    \section{Difficoltà incontrate e commenti per i docenti}
    \appendix
    \chapter{Guida utente}
    \chapter{Esercitazioni di laboratorio}
    \printbibliography[heading=bibintoc]
\end{document}
