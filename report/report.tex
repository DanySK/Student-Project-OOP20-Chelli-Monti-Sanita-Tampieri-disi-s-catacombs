\documentclass[a4paper,12pt]{report}

\usepackage[italian]{babel}
\usepackage[utf8]{inputenc}
\usepackage[T1]{fontenc}
\usepackage{pgf-umlcd}
\usepackage[style=numeric-comp]{biblatex}

\addbibresource{bibliografia.bib}
\title{DISI's catacomb report}
\author{Chelli M., Monti G., Sanità R., Tampieri E.}

\begin{document}
    \maketitle
    \tableofcontents
    \chapter{Analisi}
    \section{Requisiti}
    \subsection{Requisiti funzionali}
    % Qui si meotte la parte di analisi
    \par Il team si pone l'obiettivo di realizzare un gioco 2D roguelike,
    ovvero caratterizzato dall'esplorazione di livelli generati proceduralmente
    di un dungeon, da un gameplay a turni, da grafica tile-based e la morte
    permanente del giocatore \cite{wiki:Roguelike}, simile ai giochi Enter the
    Gungeon e Nuclear Throne.
    \par Il giocatore dovrà affrontare una serie di stanze contenenti
    nemici e oggetti fino ad arrivare a un boss finale che sarà visibile solamente una volta uccisi tutti i nemici base.
    La mappa di gioco sarà generata in modo casuale e sarà casuale anche la stanza in cui il personaggio comincerà
    la sua avventura.
    \par Inizialmente il protagonista avrà il 100\% di vita e un arma base che potrà
    cambiare con armi più avanzate che troverà nel corso
    della partità. Ad ogni colpo subito il personaggio perderà una percentuale della
    vita, se il personaggio muore la partita terminerà lasciando al possibilità all'utente di
    poter cominciare un altra partità in una nuova mappa.
    \par Il gioco può terminare anche se viene sconfitto il boss della mappa ovvero
    un nemico di dimensione maggiore rispetto a quelli incontrati nelle stanze precedenti.
    Tale boss possiederà anche abilità superiori e armi letali per il personaggio.
    \par Se il giocatore si trova in difficoltà può trovare sparse per la mappa di gioco
    alcune pozioni curative che verranno lasciate sporadicamente dopo l'uccisione dei nemici base,
    anche la quantità di vita rilasciata dalle pozioni è casuale.
    \par Esistono 2 tipi di nemici base:
    \begin{itemize}
        \item Nemici che attaccano da lontano
        \item Nemici che attaccano da vicino
    \end{itemize}
    \par Il gioco si basa sull'abilità del giocatore ma anche su una percentuale di
    fortuna nel trovare vite, armature ed armi che lo aiuteranno a sconfiggere il
    boss finale.
    \subsection{Requisiti non funzionali}
    \begin{itemize}
        \item Il gioco dovrà risultare fluido e reattivo anche su macchine con hardware non recenti..
        \item Il gioco dovrà avere una grafica e comandi chiari e intuitivi.
    \end{itemize}
    \section{Analisi e modello del dominio}
    % modello del dom
    \par Il sistema gestisce la generazione delle mappe con le varie stanze e le interazioni tra il
    personaggio e i nemici.
    \par Oltre ai nemici il personaggio potrà interagire con oggetti trovati nelle varie stanze.
    \par Il personaggio possiede principalmente una percentuale vite che aumenteranno con le pozioni
    o diminuiranno se colpiti da un nemico.
    \par I nemici potranno essere di due tipi principali, melee (che attaccano da vicino) e ranged (che attaccano da lontano)
    e si muoveranno all'interno delle stanze e avranno una vita massima. Nemici più difficili da sconfiggere,
    come i Boss avranno una maggiore vita massima e mosse speciali che verrano usate durante il combattimento.
    \par La mappa è divisa in più stanze di dimensione variabile, unite da corridoi, dove compariranno nemici
    e/o oggetti utilizzabili dal giocatore.
    Una volta sconfitti tutti i nemici presenti nella mappa il personaggio dovrà affrontare il Boss e dopo il
    combattimento il gioco terminerà.
    \par \par Gli elementi considerati nel modello sono sintetizzati in Figura 1.1


    \chapter{Design}
    \section{Architettura}
    \section{Design dettagliato}
    \subsection{Sanità}
    \par Questa parte si concentrerà sugli aspetti relativi ai vari stati che costituiscono il gioco (menu, stato di gioco, stato di fine).
     Il sistema per la creazione di questi stati è la creazione di una classe astratta State dalla quale poi verranno
     estese tutte le classi che costituiscono i diversi stati. Questa classe rappresenta lo scheletro di un qualunque
     stato ed essendo una classe astratta l’implementazione di due metodi astratti quali update e render sarà diversa da
     stato a stato in modo che la logica e la grafica di ogni stato risulti diversa.
    %UML
    \\
    \\
    \par
        \begin{tikzpicture}
            \tikzstyle{every node}=[font=\scriptsize]
            \begin{abstractclass}[text width=3.5cm]{State}{0, 0}

                \operation{+ update(delta : long)}
                \operation{+ render(g2 : Graphics2D)}
            \end{abstractclass}

            \begin{class}[text width=3.5cm]{MenuState}{-5, -5}
                \inherit{State}

                \operation{+ update(delta : long)}
                \operation{+ render(g2 : Graphics2D)}
            \end{class}

            \begin{class}[text width=3.5cm]{GameState}{0, -5}
                \inherit{State}

                \operation{+ update(delta : long)}
                \operation{+ render(g2 : Graphics2D)}
            \end{class}

            \begin{class}[text width=3.5cm]{EndGameState}{5, -5}
                \inherit{State}

                \operation{+ update(delta : long)}
                \operation{+ render(g2 : Graphics2D)}
            \end{class}

        \end{tikzpicture}
    \\
    \\
    \par Questa parte verterà sulla gestione della schermata principale e su quella di fine gioco la quale poi introdurrà nuovamente
     alla schermata di menù. Il sistema per la gestione di questi due stati è la medesima, entrambi utilizzano il pattern Strategy
     infatti è possibile modificare la grafica di queste schermate senza andare ad impattare direttamente la logica che le gestisce.
     Questa scelta è stata fatta principalmente per evitare di infrangere SRP (Single Responsibility Principle) e quindi
     di incaricare una classe sia della gestione grafica di una schermata sia della sua logica creando così una complicazione
     in un futuro cambiamento.
    %UML
    \\
    \\
    \par
        \begin{tikzpicture}
            \tikzstyle{every node}=[font=\scriptsize]
            \begin{class}[text width=3.5cm]{MenuState}{-4, -0.2}

                \operation{+ update(delta : long)}
                \operation{+ render(g2 : Graphics2D)}
            \end{class}

            \begin{interface}[text width=3.5cm]{LogicMenu}{4, 0}

                \operation{start( )}
                \operation{selectOption( )}
                \operation{getOption( )}
                \operation{isOptionStart( )}
            \end{interface}

            \aggregation{MenuState}{uses}{}{LogicMenu}

            \begin{class}[text width=3.5cm]{LogicMenuImpl}{4, -5}
                \implement{LogicMenu}

                \operation{+ start( )}
                \operation{+ selectOption( )}
                \operation{+ getOption( )}
                \operation{+ isOptionStart( )}
            \end{class}
        \end{tikzpicture}
    \\
    \\
    \par Questa parte riguarda la gestione degli asset più precisamente la creazione di strutture e il caricamento delle
    immagini all’interno di esse. Il pattern utilizzato in questo caso è Singleton permettendo che ci sia un'unica istanza
    di AssetManager accessibile globalmente senza doversi preoccupare di fornire il riferimento a chi lo richiede.
    Questa scelta è stata fatta perché il caricamento delle immagini in quelle strutture basta farla una volta sola per poi
    permetterne l’accesso a chiunque lo richieda. Lo stesso ragionamento è stato fatto per il KeyManager essendo anche esso
    di utilità a più classi evitando anche in questo caso la creazione di istanze superflue
    %UML
    \\
    \\
    \par
        \begin{tikzpicture}
            \tikzstyle{every node}=[font=\scriptsize]
            \begin{class}[text width=4cm]{AssetManager}{-3, 0}
                \attribute{- SINGLETON\_MANAGER : AssetManager}

                \operation{- AssetManager( )}
                \operation{+ getAssetManager( )}
            \end{class}

            \begin{class}[text width=5cm]{KeyManager}{3, 0}
                \attribute{- SINGLETON\_KEYMANAGER : KeyManager}

                \operation{- KeyManager( )}
                \operation{+ getKeyManager( )}
            \end{class}
        \end{tikzpicture}
    \\
    \\
    \par Questa parte di codice si occupa invece della cambiamento di stato ovvero delle transizioni che portano da uno stato
    a l’altro. In questo caso è stato utilizzata una Factory di transizione per via del fatto che la forma delle transizioni
    è la medesima  le uniche differenze riguardano gli stati di partenza e di arrivo e il messaggio che il giocatore vedrà.
    In questo non utilizzando una Factory avrei avuto tre classi in cui quasi tutto il codice veniva ripetuto e solo poche
    parti sarebbero cambiate  e così facendo avrei violato una regola base generale di buona programmazione ovvero
    DRY (Don’t Repeat Yourself).
    %UML
    \\
    \\
    \par
        \begin{tikzpicture}
            \tikzstyle{every node}=[font=\scriptsize]

            \begin{abstractclass}[text width=3.5cm]{State}{-4, 0}

                \operation{+ update(delta : long)}
                \operation{+ render(g2 : Graphics2D)}
            \end{abstractclass}

            \begin{interface}[text width=3.5cm]{TransitionFactory}{4, 0}

                \operation{transState(message : String, game : DungeonGame, state : State)}
            \end{interface}

            \begin{class}[text width=3.5cm]{TransitionFactoryImpl}{4, -5}
                \implement{TransitionFactory}

                \operation{- generalTrans(message : String, game : DungeonGame, state : State)}
                \operation{+ transState(message : String, game : DungeonGame, state : State)}
            \end{class}

                \association{TransitionFactory}{creates}{}{State}{}{}
        \end{tikzpicture}
    \\
    \\
    \chapter{Sviluppo}
    \section{Testing automatizzato}
    \section{Metodologia di lavoro}
    \par In questa parte esporrò il mio contributo per il progetto relativo alle interfacce grafiche al caricamento di risorse grafiche
    e testuali , alla gestione degli asset e delle animazioni del gioco , alla struttura del loop principale attraverso il quale si
    sviluppa il gioco e alla interazione con dispositivi esterni quali mouse e tastiera.
    \par La parte di cui mi sono maggiormente e occupato la GUI del gioco nello specifico la gestione degli stati attraverso i quali
    il gioco si sviluppa partendo dalla creazione di un generico stato\texttt{State} dal quale si estendono poi altri stati quali:
    \texttt{MenuState} , \texttt{GameState} e \texttt{EndGameState}.
    \par Un’altra parte di cui mi sono occupato riguarda le transizioni ovvero il cambiamento da uno stato a l’altro attraverso la creazione
    , per mezzo di  \texttt{TransitionFactory} , di un interfaccia che fa da ponte tra due stati. La gestione di degli stati e dei loro
    cambiamenti è affidata ad un'altra classe DungeonGame anche questa implementata da me. Questa classe si occupa di controllare
    in quale stato si trova il gioco e in quale stato dovrà trovarsi se il player muore o se finisce il gioco, in poche parole p la
    classe che gestisce la visualizzazione il funzionamento e il susseguirsi degli stati di gioco. Inoltre fornisce informazioni
    riguardanti la finestra principale del gioco creata dalla classe \texttt{MainWindow} di mia implementazione anche se la totalità delle
    informazioni riguardanti la finestra è disponibile a chi  ne facesse uso nella mia classe \texttt{GameConfiguration}.
    \par Il \texttt{DungeonGame} estende la classe \texttt{Game} che contiene i metodi necessari all’inizializzazione della finestra di gioco,
    il render degli aspetti di base ma soprattutto si occupa del loop principale del gioco.
    \par Un’altra classe di del mio codice è la gestione degli input di dispositivi esterni quale la tastiera \texttt{KeyManager} che
    gestisce i vari input da tastiera.
    \par Una altra parte di mia responsabilità è il caricamento di risorse come si può vedere dal package \texttt{catacombs/ui/utils (FontUtils, ImageLoader e ImageRotator)}
    e dall’\texttt{AssetManager} che fa uso di un’altra classe, \texttt{GameSheet} che si occupa di ritagliare le immagini in immagini più piccole
    in modo da separare i singoli frame di movimento dei personaggi. Tali frame verranno poi utilizzati dalla classe \texttt{Animation} che
    provvede a sequenziarli in modo da farli risultare parte di un unico fluido movimento.
    \par Usufruendo dell funzinalità di branching che Git offre è stato facile rimanere aggiornato sugli sviluppi del progetto anche non
    relativi alle mie implenatazioni. la metofdologia utilizzata cosnisteva nell' utilizzo di un master principale da aggiornale solo con
    il codice ultimato e più volte controllato da me e dai miei compagni.ogni parte di codice da me realizzata si sviluppava all'interno di branch
    in modo da non sporcare il codice definitivo contenuto nel master. solo dopo aver ultimato ogni modifica ed aver ottenuto l'approvazione dei miei colleghi
    il branch veniva mergiato con il master.
    La suddivisione dei lavori precedentemente stabilita veniva rispecchiata nella spartizione degli issue ovvero degli obbiettivi parziali da risolvere che uniti
    andavano a comporre una parte sostanziosa di codice.
    \section{Note di sviluppo}
    \begin{itemize}
            \item Uso di Optional per la gestione degli eventuali file non trovati
            \item Uso di basilari lambda expressions nei test automatizzati
    \end{itemize}
    \chapter{Commenti finali}
    \section{Autovalutazione e lavori futuri}
    \section{Difficoltà incontrate e commenti per i docenti}
    \appendix
    \chapter{Guida utente}
    \par Il gioco comincerà con una schermata di menu che permette di cambiare l'opzione che si desidera scegliere attraverso l'uso
     dei tasti direzionali o  WASD, per selezionare la chiusura della finestra o l'avvio del gioco.
    \par Una volta che l'indicatore di selezione si sarà spostato sull'opzione desiderata è sufficiente premere il tasto invio per confermarla.
    \par Iniziata la partita, il personaggio si potrà spostare la mappa utilizzando i tasti indicati in precedenza.
    \par Il giocatore potrà sparare ai nemici per eliminarli usando il tasto spazio, sia tenendolo premuto che premendolo ripetutamente,
     e per poter mirare al nemico basterà rivolgere il corpo del giocatore nella direzione desiderata (e.g se un nemico si trova alla
     proria destra, l'utente dovrà premere D e poi premere la barra dello spazio).
     \par Nella mappa sono presenti alcuni oggetti utili, che possono essere raccolti camminandoci sopra.
     \par Nello specifico, sono presenti delle pozioni (rappresentate da un'ampolla) per aumentare la propria vita e delle armi. Queste ultime possono essere:
     \begin{itemize}
         \item Una pistola, che è l'arma con cui il giocatore inizia il gioco
         \item Un fucile automatico che, rispetto alla pistola, permette di sparare più frequentemente.
     \end{itemize}
     \par Dopo aver ucciso un nemico, è possibile che questo lasci cadere un'arma o una pozione.
    \chapter{Esercitazioni di laboratorio}
    \printbibliography[heading=bibintoc]
\end{document}
